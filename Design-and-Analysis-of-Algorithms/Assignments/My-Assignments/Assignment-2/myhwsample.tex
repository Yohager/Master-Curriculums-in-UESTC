\documentclass{myhw}
\usepackage{ctex}
\linespread{1.05}        % Palatino needs more leading (space between lines)
\usepackage{extarrows}
\usepackage{mathrsfs}
\usepackage{braket}
\usepackage{enumerate}
\usepackage{tikz}
\usepackage[ruled]{algorithm2e}
\usepackage{tikz-network}
\titleformat{\section}[runin]{\sffamily\bfseries}{}{}{}[]
\titleformat{\subsection}[runin]{\sffamily\bfseries}{}{}{}[]
\renewcommand{\exname}{Exercise }
\renewcommand{\subexcounter}{(\alph{homeworkSectionCounter})}
\newcommand{\id}{\text{Id}}
\newcommand{\tr}{\text{Tr}}
\newcommand{\rib}{\text{Rib}}

\title{Design and Analysis of Algorithms}

\begin{document}
\begin{center}
\noindent{\Large \heiti 2020年算法设计与分析 \textbf{Assignment-2}}

\vspace{0.5cm}

Due: Nov 03, 2020

\vspace{0.5cm}
\noindent {\large 郭宇航  202021080728}
\end{center}
\begin{homeworkProblem}
请简述证明一个问题是NP完全问题的主要步骤。
\end{homeworkProblem}
\begin{solution}
证明分为两步:\\
(1)首先证明这个问题是一个NP问题;\\
(2)其次证明一个NP完全问题都可以归约到这个问题。\\
这样就完成了一个问题是NP完全问题的证明。
\end{solution}
\begin{homeworkProblem}
请证明:如果我们可以在多项式时间内给出判定一个图是否存在哈密顿圈,则我们可以在多项式时间内找到一个图的哈密顿圈。(如果存在的话)
\end{homeworkProblem}

\begin{solution}
假设在多项式时间内判断一个图是否存在哈密顿圈的问题记为问题$X$,将多项式时间内找到一个图的哈密顿圈问题记为问题$Y$,那么下面的目标是在多项式时间内将问题$Y$归约到问题$X$:$Y\leq_p X$.
\begin{itemize}
\item 对于给定的图,首先调用问题$X$的解判断是否存在哈密顿圈,如果不存在则直接返回不存在方案,否则继续。
\item 在给定的图中随机选择一条边,删除后调用问题$X$的解判断剩下的图是否存在一个哈密顿圈,如果不存在则这条边是原图的一个哈密顿圈的中一条边,将这个边添加入方案集合中,如果剩余图中仍然存在一个哈密顿圈则跳过这条边。
\item 重复上述的操作直到遍历完整个给定的图的所有边,返回一个哈密顿图的方案。
\end{itemize}

\end{solution}
\begin{homeworkProblem}
最长路径问题:给定简单无向图中是否存在长度大于等于$k$的简单路径?请证明这个问题是NP完全问题。
\end{homeworkProblem}

\begin{solution}
(1)首先考虑证明最长路径问题是一个NP问题:显然给出一个问题的实例一条简单路径,我们可以在多项式时间内验证这个实例路径的长度是否大于等于$k$. \\
(2) 下面证明一个NP完全问题可以归约到最长路径问题,这里证明图的哈密尔顿圈问题可以归约到最长路径问题,即:HAMILTON-CYCYLE $\leq_p$ LONGEST-PATH. 归约流程为:(1)首先给出一个哈密顿图的实例$G$,我们对应的构建一个最长路径的实例$(G',K)$,其中$G=G', K=|V|-1$. 显然根据图$G$中存在哈密顿圈,我们一定可以在图$G'$中找到一个长度为$K$的平凡路径。
\end{solution}

\begin{homeworkProblem}
图的匹配是一个边集,其中任意两条边都没有公共的顶点。请设计近似算法求解图的最大匹配问题,使得所得匹配集包含的顶点数超过最优解的$1/2$.
\end{homeworkProblem}

\begin{solution}
简单的贪心算法如算法\ref{1}所示:根据上面的贪心算法我们不难发现我们可以自然得到一个点覆盖$|S|=2|M|$,假设$M^\ast$表示一个最大的图匹配解,显然我们可以知道前面得到的这个点覆盖结果$S\geq |M^\ast|$.(对于给定的一个图,假设$M$表示任意的一个图匹配解,$S$表示任意的一个点覆盖解,一定有$|S|\geq |M|$)从而我们可以得到:
\[
|M|=\frac{1}{2}|S|\geq \frac{1}{2}|M^\ast|
\]
从而我们找到了一个满足$1/2$近似的图匹配近似算法。

\begin{algorithm}[H]
\caption{GREEDY-GRAPH-MATCHING ALGORITHM}
\label{1}
\LinesNumbered
\KwIn{graph $G=(V,E)$}
\KwOut{graph-matching: $M$}
\textbf{Initialize} $M\rightarrow \emptyset$, $E'\rightarrow E$\;
\While {$E'\neq \emptyset$}{
Let $(u,v)\in E'$ be an arbitrary edge\;
$M\rightarrow M\cup \{(u,v)\}$.\;
Delete from $E'$ all edges incident to either $u$ or $v$.
}
\
\Return $M$ 
\end{algorithm}

\end{solution}
\end{document}
