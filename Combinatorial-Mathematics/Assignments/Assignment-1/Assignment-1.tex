\documentclass[a4paper]{ctexart}
\pagestyle{plain}
\usepackage{amsmath}
\usepackage{amssymb}
\usepackage{graphicx}
\title{\textbf{Assigment 1 for Combinational Mathmatics}}
\date{2020.9.18}
\author{Guo Yuhang \\ 202021080728
}
\begin{document}
	\maketitle
	\section*{Exercises 1}
	2. 求在1000-9999各位数字都不相同而且由奇数构成的整数的个数。
	
	\textbf{Soultion}: 思路是先取个位数,个位数只能取$1,3,5,7,9$,其次再去取千位,由于千位存在不能为0的约束同时各位数字不相同,所以千位存在8种取法,最后对无约束的百位和十位进行数字选择,分别有8种和7种选择方式。因此最终的结果表示为:
	$$
	5\times 8\times 8 \times 7=2240
	$$
	
	4. 10个人坐在一排看戏有多少种就坐方式?如果其中两个人不愿意坐在一起又有多少种就坐方式?
	
	\textbf{Solution}: 第一个问题是一个简单的线排列问题,结果为:$P(10,10)$.\\
	第二个问题:存在两种思路:第一种直接计算,首先先排9个人,最后一个人由于有约束只存在8个位置可以安排,因此最终的结果可以表示为:$P(9,9)\times 8$. 第二种思路可以先不考虑约束排,最后减去不满足条件的解,也就是$P(10,10)-2\times P(9,9)$.
	
	5. 10个人围成圆桌就坐,其中有两个人不愿意坐在一起,请问共有多少种就坐方式?
	
	\textbf{Solution}: 这个问题是圆排列问题,也可以从两个思路出发:第一种直接计算,先对9个人进行圆排列,最终的第10个人根据约束只存在7个位置可以就坐,因此结果表示为:$P(9,9)/9\times 7$. 第二种思路:先计算无约束条件的排列情况:$P(10,10)$, 然后再计算不满足条件的情况:使用绑定的计算方法,$2\times P(9,9)/9$. 最终结果为:$P(10,10)-2\times P(9,9)/9$.
	
	6. 6男6女围圆桌交替就坐存在多少种就坐方式?
	
	\textbf{Solution}: 先排女生,$P(6,6)/6$,再将6个男生排入,第一个男生有6种安排方案,第二个有5种,依次类推可以得到最终结果为:$P(6,6)/6\times 6\times 5\times 4\times 3\times 2\times 1$.
	
	7. 由1,2,3,4,5这5个数字能够组成多少个没有重复的数字?不能被5整除同时大于20000的五位数?
	
	\textbf{Soultion}: 分情况讨论,首先考虑万位排2,3,4的情况下,个位只能3个数字可选,剩下的位次随意排,因此这种情况的结果可以写为:$3\times 3\times P(3,3)$; 第二种情况,万位排5,此时剩下的4位随意排,结果为:$1\times P(4,4)$. 因此最终的答案可以写为:$3\times 3\times P(3,3)+1\times P(4,4)=78$.
	
	10. 在1000-9999的整数,有多少个整数包含数字3一次?有多少整数包含数字3?又有多少个数字包含3个7?
	
	\textbf{Solution}: (1). 4位数,数字3先选,3如果在千位,那么一共有:$1\times 9^3$;如果3在百十个位,那么一共有:$3\times 8\times 9^2$. 合计共有:$9^3+3\times 8\times 9^2=2673$个数字包含数字3一次。 (2) 考虑4位数存在包含数字3的情况一共有4种:包含4个3:1种;包含3个3:$1\times 8+3\times 9=35$种;包含2个3:$1\times 3\times 9^2 + 8\times 3\times 9=459$;包含一个3:$1944$种(刚刚已经求解过)。因此包含数字3的整数个数位:$1+35+459+2673=3168$. 最后一个问题包含3个7的问题:可以分为第一个位置为7或者不为7的情况,为7的情况就是$1\times 3\times 9=27$个,第一个位置不为7的情况:$8$个整数,共计为$27+8=35$个整数中包含3个7.
	
	11. 单词MISSISSIPPI中的字母有多少种不同的排列方式?如果两个S不相邻又有多少种排列方法?
	
	\textbf{Soultion}: 这个是一个重排列问题,首先我们可以将重集写出来:$B=\{1\cdot M, 4\cdot I, 4\cdot S,2\cdot P\}$. 那么这个重集$B$做全排列的结果为:$\frac{11!}{4!\times 4!\times 2!\times 1!}=34650$种不同的排列方式。如果两个$S$不相邻的情况,那么我们可以先对除了S之外的其他7个字母先做全排列,然后再将S字母一一插入其中,最终结果可以表示为:$\frac{7!}{4!\times 2!\times 1!}\times \binom{8}{4}=7350$.
	
	13. 求解方程$x_1+x_2+\times +x_n=r$的正整数解的个数。
	
	\textbf{Soultion}: 之前解过类似的题目,形如:$x_1+x_2+\cdots +x_n=r$的非负整数解的个数?这个问题理解为一个重排列问题,现在可以将13题改写为上面的形式;$(x_1-1)+(x_2-1)+\cdots + (x_n-1)=r-n$的非负整数解的个数,等于重排列:$F(n,r-n)=F(r-1,r-n)=F(r-1,n-1)$. 
	
	14. 在1-10000中,有多少整数,它的数字之和等于5?又有多少数字之和小于5的整数?
	
	\textbf{Solution}: (1) 考虑数字之和等于5的情况一共存在7种情况,第一种一个5和4个0,共有$4$种;第二种情况一个4一个1和3个0,共有$12$种;第三种情况一个3一个2和3个0,共有:$12$种;第四种情况1个3和2个1和2个0,共有$12$种,第五种情况2个2和1个1以及2个0,共有$12$种;第六种情况1个2,3个1以及1个0,共有$4$种;最后一种情况5个1不满足条件,共有$0$种,因此根据加法规则共有:$4+12+12+12+12+4=56$种。
	(2) 考虑数字之和小于5个情况,则分为和等于1,2,3,4进行讨论,计算方式与(1)种的方式相似:和为1的情况:$5$种;和为2的情况:$10$种;和为3的情况:$20$种;和为4的情况:$35$种。综上数字之和小于5的整数的个数为:$5+10+20+35=70$种。\\
	因此问题14的解为(1)共有56个整数数字之和等于5;(2)共有70个整数的数字之和小于5.
	
	16. 从整数$1,2,\cdots,1000$中选取三个数使得他们的和是4的倍数,求这样的选法存在多少种?
	
	\textbf{Solution}: 根据余数可以将这1000个整数分为四类,分别是除以4余数为0,1,2,3的情况。我们写为4个集合:$A_1,A_2,A_3,A_4$. 其中:
	\[ 
	A_1 = \{1,5,9,\cdots,997 \}\\
	A_2 = \{2,6,10,\cdots,998\}\\
	A_3 = \{3,7,11,\cdots,999\}\\
	A_4 = \{4,8,12,\cdots,1000\}\\
	 \]
	 我们不难求解得到:$|A_1|=\lceil \frac{997}{4} \rceil=250, |A_2|=\lceil \frac{998}{4} \rceil=250,|A_3|=\lceil \frac{999}{4} \rceil=250,|A_4|=\lceil \frac{1000}{4} \rceil=250$. \\
	 要求取出的3个数能够被4整除,存在的情况有以下5种:三个数余数均为0;两个数余数为2一个数余数为0;两个数余数为1一个数余数为2;三个数余数分别为0,1,3;两个余数为3一个余数为2;可以分别计算各类情况下的结果,最终结果表示为:$\binom{250}{3}+250^3+3\times \binom{250}{1}\times \binom{250}{2}$.
	 
	 19. 使用组合分析的方法证明恒等式:
	 \[ \sum_{k=0}^n \binom{n}{k}=2^n \]
	 
	 \textbf{Solution}: 首先等式的左边我们可以理解为从$n$个物品取0个,1个,2个……一直到取$n$个。这也就表示着我们将这$n$个物品取或者不取的情况全部都考虑了一遍,而等式的右边$2^n$我们可以理解为对于这$n$个物品,每一个都存在取或者不取的两种情况,也就可以写为$2^n$. 因此等式两边表示的均为$n$个物品中选择任意个物品的所有情况的表示。
	 
	 24. 证明下面的两个恒等式:
	 \[
	 \begin{aligned}
	 (1) & \sum_{k=0}^{m}\binom{n-k}{m-k}=\binom{n+1}{m}\\
	 (2) & \sum_{k=m}^n\binom{k}{m}\binom{n}{k} = \binom{n}{m} 2^{n-m}
	 \end{aligned}
	 \]
	 
	 \textbf{Solution}: \\
	 对于问题(1)等式左边反复使用Pascal公式:
	 \[ 
	 \begin{aligned}
	  &\sum_{k=0}^{m}\binom{n-k}{m-k}\\
	  =& \binom{n}{m} +\binom{n-1}{m-1} + \cdots +\binom{n-m+1}{1}+\binom{n-m}{0}\\
	  =& \binom{n}{m} +\binom{n-1}{m-1} + \cdots +\binom{n-m+1}{1}+\binom{n-m}{0} + 0\\
	  =& \binom{n}{m} +\binom{n-1}{m-1} + \cdots +\binom{n-m+1}{1}+\binom{n-m}{0} + \binom{n-m}{-1}\\
	  =& \binom{n}{m} +\binom{n-1}{m-1} + \cdots +\binom{n-m+1}{1}+\binom{n-m+1}{0}\\
	  =& \cdots \\
	  =& \binom{n+1}{m}
	 \end{aligned}
	  \]
	  等于等式右边$\binom{n+1}{m}$,由此等式得证。\\
	  对于问题(2)首先对等式左边进行考虑:
	  \[ 
	  \begin{aligned}
	  &\sum_{k=m}^{n} \binom{k}{m}\binom{n}{k}\\
	  =&\sum_{k=m}^{n} \frac{k!}{m!(k-m)!} \frac{n!}{k!(n-k)!}\\
	  =&\sum_{k=m}^{n} \frac{n!}{(n-k)! m!(k-m)!}\\
	  =&\sum_{k=m}^{n} \frac{n! (n-m)!}{(n-k)! m!(k-m)!(n-m)!}\\
	  =& \sum_{k=m}^{n} \frac{n!}{m!(n-m)!}\frac{(n-m)!}{(n-k)!(k-m)!}\\
	  =& \sum_{k=m}^{n} \binom{n}{m} \binom{n-m}{k-m}\\
	  =&\binom{n}{m}\sum_{k=m}^{n} \binom{n-m}{k-m}\\
	  =&\binom{n}{m} 2^{n-m}\\
	  \end{aligned}
	   \]
	  等于等式右边:$\binom{n}{m}2^{n-m}$. 由此等式得证。
	  
	 26. 下图\ref{1}表示了一张城市平面图,图中的直线表示街道,直线的交点表示街道的交叉路口。证明从交叉路口$S(0,0)$到交叉路口$T(m,n)$共有$\binom{m+n}{n}$条不同的路径可以走。
	 \begin{figure}[t]
	 	\centering
	 	\includegraphics[width=0.6\textwidth]{problem-26.png}
	 	\caption{城市平面图}
	 	\label{1}
	 \end{figure}
 
    \textbf{Solution}:
	考虑这个问题,首先我们知道这里的所有路径都应该表示的是从$S(0,0)$点出发要么向上要么向右最终到达$T(m,n)$点的路径的所有排列情况。我们将这个问题想象为一个重排列问题,首先定义重集$B=\{n \times u,m\times r\}$. 其中$u$表示up, $r$表示right. 那么由这个重集所组成的所有的重排列的结果可以表示为:
	\[ \frac{m+n!}{m!n!} =\binom{m+n}{n}\]
	由此上面给出的结论得证。
 \end{document}