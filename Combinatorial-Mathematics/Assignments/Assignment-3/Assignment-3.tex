\documentclass[a4paper]{ctexart}
\pagestyle{plain}
\usepackage{amsmath}
\usepackage{amssymb}
\usepackage{graphicx}
%\usepackage{tikz}
%\usetikzlibrary{patterns}
\title{\textbf{Assigment 3 for Combinational Mathmatics}}
\date{2020.9.28}
\author{Guo Yuhang \\ 202021080728
}

\begin{document}
\maketitle
\section*{Exercise 3}
%3,5,6,7,8,9,10,11,12
3. 在边长为1的正三角形中任选5个点,证明存在两点,其距离不超过1/2.\\
\textbf{Solution:} 将这个边长为1的正三角形取三条边的中点并进行连接。我们会发现构成了四个边长为1/2的小正三角形。将四个小正三角形看作盒子,问题转化为任取5个点放入这4个盒子中。根据鸽笼原理可知必定有2个点在同一个盒子中,也就是说任选的两个点一定在同一个小正三角形中,而在这个小正三角形中,两点最远距离为1/2. 因此结论得证。

5. 在图5-9中,每个方格着红色或蓝色,证明至少存在两列有相同的着色。\\
\textbf{Solution:} 考虑任意一列的方格的着色可能的情况:红色+红色;红色+蓝色;蓝色+红色;蓝色+蓝色。将这4中着色方案看作4个盒子,将5列格子放入这4个盒子。根据鸽笼原理可知,至少有两列放在了同一个盒子中,也就是说至少有两列有相同的着色方案。因此结论得证。

6. 任给5个整数,则必定能从中选出3个使得他们的和能够被3整除。\\
\textbf{Solution:} 考虑任何一个整数除以3之后的余数可能为$0,1,2$. 首先考虑第一种情况:如果这5个数中存在3个及以上的数除以3的余数相同,则显然可以找到其中的3个数使得他们的和能够被3整除;另外一种情况,假设这5个数中最多只有2个及以下的数除以3的余数相同,在这种情况下存在的可能性有:$00112,00122,01122$这三种情况,不难发现这三种情况下我们均可以找到3个数使得他们的和能够被3整除。综上两种情况,结论成立。

7. 一个学生打算使用37天共计60学时自学一本书,他计划每天至少自学1个小时。证明无论他怎样安排自学时间表,必然存在相继的若干天在这些天中其自学的总时数恰好为13学时(假设每天自学的学时数为整数)。\\
\textbf{Solution:} 设$a_i$表示该学生前$i$天学习的总时数,其中$i=1,2,\cdots,37$. 于是$a_1,a_2,\cdots,a_{37}$是一个严格递增的序列且我们可以知道:$a_1\geq 1,a_37\leq 60$. 于是我们可以知道:$a_1+13,a_2+13,\cdots,a_{37}+13$也是一个严格递增序列且$a_37\leq 73$. 于是我们知道:
\[
a_1,a_2,\cdots,a_{37},a_1+13,a_2+13,\cdots,a_{37}+13
\]
这74个数都在1-73之间,那么根据鸽笼原理可知一定存在两个数相等,假设$a_k$和$a_l+13$相等,则我们可以得到:
\[
a_k-a_l=13
\]
因此我们可以知道在第$k$天到第$l$天之间这个学生一共学习的总时数等于13学时。

8. 已知$n$个正整数$a_1,a_2,\cdots,a_n$,证明在这$n$个数中总可以选出两个数使得这两个数的和或者差能够被$n$整除。\\
\textbf{Solution:} 对于任意的$a_i$,其$a_i \% n \in [0,n-1]$. 考虑第一种情况,存在一个数$a_i\%n=k$同时$a_j\%n=k$. 则这两个数的差可以被$n$整除,结论成立;下面考虑第二种情况,如果不存在一个数对$n$取模为$k$同时另一个数对$n$取模为$k$. 即所有的取模的结果都不相同,那么对于$n$个数一定存在一个$k$使得$a_i\%n=k$同时$a_j\%n=n-k$. 则在这种情况下,两个数之和可以被$n$整除。综上所述,在$n$个正整数种一定可以在这$n$个数种找到两个数使得两个数的和或者差能够被$n$整除。

9. 设$a_1,a_2,\cdots,a_n$是$1,2,\cdots,n$的一个排列。证明:如果$n$是奇数,则乘积$(a_1-1)(a_2-2)\cdots(a_n-n)$是一个偶数。\\
\textbf{Solution:} 当$n$为奇数时,我们不难发现$1,2,\cdots,n$中共有$\frac{n-1}{2}$个偶数同时$\frac{n+1}{2}$个奇数。考虑$\prod_{i=1}^n (a_i-i)$,如果$a_i$和$i$的奇偶性相同则$a_i-i$一定为偶数,反之为奇数。下面我们使用反证法证明,假设乘积$\prod_{i=1}^n(a_i-i)$时一个奇数,则被乘的每一项都应该是奇数。考虑奇偶数的搭配,对于$a_i$存在$\frac{n-1}{2}$个偶数,需要在$i$中找出$\frac{n-1}{2}$个奇数与之对应,反之也相同,则最终会多余两个奇数,这两个奇数差一定为偶数,从而$\prod_{i=1}^n (a_i-i)$一定为偶数,与假设矛盾,因此结论得证。

10. 证明:在任意52个整数中,必存在两个数,其和或差能被100整除。\\
\textbf{Solution:} 考虑任意的整数除以100的余数可以为$0,1,2,\cdots,99$. 考虑第一种情况,存在一个数$a_i$对100取模的结果为$k$同时$a_j$对于100取模的结果为$100-k$. 那么这种情况下$a_i+a_j$的结果一定能够被100整除。另一种情况下考虑如果不存在这样的组合结果存在$\{1,99\},\cdots,\{49,51\}$共有49种,那么一共就存在$49+1+1$种可能的余数结果,而共有52个整数存在,利用鸽笼原理我们会发现一定会有两个整数属于同一类,也就是说必定存在两个数$a_i,a_j$的余数相等(两数之差可以被100整除);或者这两个数的余数求和等于100,从而两数之和能够被100整除。综上所述,在这任意的52个整数中必定存在两个数其和或者差能够被100整除。

11. 证明$R(4,4)\leq 18$.\\
\textbf{Solution:} 根据Ramsey定理,当$a,b\geq 2$时,$R(a,b)$是一个有限数且有$R(a,b)\leq R(a-1,b)+R(a,b-1)$. 那么我们可以得到$R(4,4)\leq R(3,4)+R(4,3)$. 而根据准确计算我们可以知道$R(3,4)=R(4,3)=9$.因此$R(4,4)\leq 18$. 

12. 证明:$R(a_1,a_2,\cdots,a_m)\leq R(a_1-1,a_2,\cdots,a_m)+R(a_1,a_2-1,a_3,\cdots,a_m)+\cdots+R(a_1,a_2,\cdots,a_m-1)$.\\
\textbf{Solution:} 设$R_i=R(a_1,a_2,\cdots,a_i-1,\cdots,a_n),i=1,2,\cdots,n$, 我们记$N=\sum_{i=1}^n R_i$. 考虑$N$个顶点的完全图,使用$n$中颜色$c_i(i=1,2,\cdots,n)$对完全图的边进行染色。在$N$个点中任取一点$P$, 由$P$连出$N-1$条边。则有:
\[
\begin{split}
N-1&=R_1+R_2+\cdots+R_n-1\\
&\geq R_1+R_2+\cdots +R_n-(n-1)\\
&=R_1+R_2+\cdots +R_n-n+1
\end{split}
\]
现在存在$c_1,c_2,\cdots,c_n$个盒子,根据鸽笼原理我们可以知道至少存在一个$i(1\leq i\leq n)$使得这$N-1$条边中染成$c_i$颜色的边数至少是$R_i$条。考虑其中染成$c_i$颜色的$R_i$条边,这$R_i$条边连接到另外的$R_i$个顶点,而$R_i=R(a_1,a_2,\cdots,a_{i-1},a_i-1,\cdots,a_n)$, 因此这$R_i$个顶点之间的连线要么有$c_1$个纯$a_1$角形;或者有$c_2$个纯$a_2$角形……或者有$c_i$个纯$a_i-1$角形(而我们知道点$P$是和这$a_i-1$个顶点所连的边都是$c_i$颜色的,从而这$a_i$个顶点组成了一个纯$a_i$角形,也就是说存在一个$c_i$颜色的纯$a_i$角形)……或者存在$c_n$个纯$a_n$角形. 而$R(a_1,a_2\cdots,a_n)$是保证出现上述情况之一的最小正整数,从而有:
\[
R(a_1,a_2,\cdots,a_n)\leq R(a_1-1,a_2,\cdots,a_n)+R(a_1,a_2-1,\cdots,a_n)+\cdots+R(a_1,a_2,\cdots,a_n-1)
\]
综上所述,原命题成立。
\end{document}